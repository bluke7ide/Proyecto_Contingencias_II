\documentclass{article}
\usepackage{amsmath}
\usepackage{geometry}
\geometry{margin=1in}

\title{Nota Técnica del Seguro de Vida Individual}
\date{}

\begin{document}
	
	\maketitle
	
	\section*{1. Declaración Actuarial}
	\textit{``(Nombre del actuario), con número de identificación \_\_\_\_\_\_\_, hago constar bajo mi responsabilidad profesional, que la metodología para la determinación de la prima, y demás elementos técnicos considerados en la presente nota técnica, se apegan a lo previsto en la Ley Reguladora del Mercado de Seguros y la normativa vigente emitida por el Consejo Nacional de Supervisión del Sistema Financiero.''}
	
	\section*{2. Características del Producto}
	\begin{itemize}
		\item \textbf{Nombre del Producto}: Seguro de Cuidados a Largo Plazo (no se a definido).
		\item \textbf{Categoría, Ramo y Línea}: Seguro Personal - Salud - Cuidados a Largo Plazo con beneficios por invalidez y fallecimiento.
		\item \textbf{Modalidad de Contratación}: Individual.
		\item \textbf{Temporalidad del Producto}: Vitalicio (cobertura hasta el fallecimiento).
		\item \textbf{Tipo de Contrato}: Contrato de adhesión.
		\item \textbf{Renovabilidad}: Renovable (no estoy seguro, como es vitalicio no sé si se considera renovable)
		\item \textbf{Moneda}: Colones costarricenses.
		\item \textbf{Canal de Distribución}: (No sé).
	\end{itemize}
	
	\section*{3. Descripción de las Coberturas}
	\begin{itemize}
		\item \textbf{Cobertura Básica}: Condición de discapacidad Modera, severa y profunda del asegurado y fallecimiento del asegurado, con una suma asegurada pagadera a final de año al beneficiario en caso de permanecer en alguno de los estados en el momento de pago.
		
		\item \textbf{Coberturas Adicionales}: Será que la muerte se considera como un adicional? Porque esto es más un seguro de salud y no de vida.
		\item \textbf{Cobertura de Servicios}: No sé si establecemos asistencia médica o se indica solo como ayuda económica.
	\end{itemize}
	
	\section*{4. Hipótesis Técnicas y Estadísticas}
	\begin{itemize}
		\item \textbf{Hipótesis Demográficas}: En el caso de los seguros personales, se deberán indicar e incluir las hipótesis demográficas como son tablas de mortalidad, de morbilidad, incapacidad, así como tablas de frecuencia, índice de siniestralidad o cualquier otra información técnica o estadística que se utilizará para el cálculo de las primas de riesgo y activo o pasivo por la cobertura restante, según lo dispuesto en el Reglamento sobre la Solvencia de Entidades de Seguros y Reaseguros.
		
	\item	Se deberá indicar cualquier aspecto relevante sobre la modificación, depuración y transformación que haya realizado a los datos originales de la estadística. Como el ajuste que se realizó a la probabilidades de transición de degradación.
	\end{itemize}
	
	\section*{5. Hipótesis Financieras}
	\begin{itemize}
		\item \textbf{Tasa de Interés Técnico}: 5\% (Se debe justificar... pero como que espera que haya?) además debe de justificar supuestos de inflación
		\item \textbf{Justificación}: Las indicaciones dicen: conforme a los principios establecidos para estos efectos, en los estándares de práctica actuarial y la normativa vigente... pero no me queda claro.
	\end{itemize}
	
	\section*{6. Procedimientos para el Cálculo de la Prima de Riesgo}
	\begin{itemize}
		\item \textbf{Fórmulas de primas de riesgo:}: Descripción lo más explícita posible para el cálculo de la prima (sin considerar ganancias o utilidades)
		\item \textbf{Recargo de Seguridad}: Un 3\% adicional sobre la prima de riesgo (por decir algo, pero no se que nos puede llegar a saber cuanto de recargo)
		\item \textbf{Deducibles y Coaseguros}: No aplica para esta cobertura (creo).
		\item \textbf{Recargos y descuentos basados en el riesgo:} No se ha conversado pero creo que no es necesario, el modificar la prima según el estado.
		\item \textbf{Punto e:} supongo que es como descuentos o beneficios por presentar poca sinestrialidad. 
	\end{itemize}
	
	\section*{7. Procedimientos para la Prima Comercial}
	\begin{itemize}
		\item \textbf{Fórmulas de prima comercial o de tarifa:}: 
		\[
		P_C = P + G + A + M
		\]
		donde:
		\begin{itemize}
			\item $G$: Gastos administrativos (5\% de la prima de riesgo).
			\item $A$: Costos de adquisición (3\% de la prima de riesgo).
			\item $M$: Margen de utilidad (2\% de la prima de riesgo).
		\end{itemize}
	No estoy seguro si es calcular el valor de la prima explícitamente pero considerando todos los gastos y utilidad.
		\item \textbf{Recargos y Descuentos}: No sé
		\item  \textbf{Recibo de prima:} Se deberán indicar los otros cargos y gastos fiscales repercutibles al tomador o asegurado asociados al seguro y que forman parte del recibo de prima.
	\end{itemize}
	
	\section*{8. Participación de Beneficios}
	Este seguro no contempla una participación en beneficios para el asegurado.
	
	\section*{9. Modelo de Medición del Pasivo por la Cobertura Restante}
	\textbf{Modelo de Medición}: Enfoque de asignación de la prima, cumpliendo con los criterios de  para la NIIF 17.
	
	\section*{10. Pérdida Máxima Probable}
	Para este producto no aplica, dado que se establece una suma asegurada máxima específica.
	
	\section*{11. Información Complementaria}
	\textbf{Revisión y Actualización}: 
	
\end{document}
